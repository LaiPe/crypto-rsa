\documentclass[10pt,a4paper,french]{article}
\author{par Léo Peyronnet}
\title{Code RSA}
\date{Décembre 2022}

\usepackage[utf8]{inputenc}
\usepackage[T1]{fontenc}

\usepackage{babel}
\usepackage{listings}
\usepackage{amsfonts}
\usepackage{amsmath}
\usepackage{amssymb}
\usepackage{amsthm}
\usepackage{mathtools}
\usepackage{tabto}
\usepackage{listings}
\usepackage{amssymb}
\usepackage{tabto}
\usepackage{xcolor}



\newcounter{exercice}[section]
\newenvironment{exercice}[1][]{\refstepcounter{exercice}\par\medskip
   \noindent \textbf{Exercice~\theexercice #1} \rmfamily}{\medskip}


\definecolor{ao(english)}{rgb}{0.0, 0.5, 0.0}
\lstset{
  extendedchars=true,
  framexleftmargin=16pt,
  framextopmargin=3pt,
  framexbottommargin=6pt,
  frame=tb,
  commentstyle=\color{ao(english)},
  breaklines=true,
  language=Python,
  literate=
  {²}{{\textsuperscript{2}}}1
  {⁴}{{\textsuperscript{4}}}1
  {⁶}{{\textsuperscript{6}}}1
  {⁸}{{\textsuperscript{8}}}1
  {€}{{\euro{}}}1
  {é}{{\'e}}1
  {è}{{\`{e}}}1
  {ê}{{\^{e}}}1
  {ë}{{\¨{e}}}1
  {É}{{\'{E}}}1
  {Ê}{{\^{E}}}1
  {û}{{\^{u}}}1
  {ù}{{\`{u}}}1
  {â}{{\^{a}}}1
  {à}{{\`{a}}}1
  {á}{{\'{a}}}1
  {ã}{{\~{a}}}1
  {Á}{{\'{A}}}1
  {Â}{{\^{A}}}1
  {Ã}{{\~{A}}}1
  {ç}{{\c{c}}}1
  {Ç}{{\c{C}}}1
  {õ}{{\~{o}}}1
  {ó}{{\'{o}}}1
  {ô}{{\^{o}}}1
  {Õ}{{\~{O}}}1
  {Ó}{{\'{O}}}1
  {Ô}{{\^{O}}}1
  {î}{{\^{i}}}1
  {Î}{{\^{I}}}1
  {í}{{\'{i}}}1
  {Í}{{\~{Í}}}1,
}

\begin{document}
\maketitle
\section{Réponses Exercices}
\begin{exercice}
\\$N=391$, $E=151$ et $D=7$
\begin{enumerate}
\item Message reçu et crypté: $C=17$\\
Soit $M$ le message tel qu'envoyé (non crypté), alors:\\
$M=C^{D}[N]=17^{7}[391]=204$.

\item On sait que $N=p\times q$ avec $p,q$ deux nombres premiers. On a donc:\\
$391=p\times q=17\times23$ \textit{(résultat obtenu avec le programme cf \ref{exo1})}\\
Nous pouvons donc déduire $\varphi(N)$:\\
$\varphi(N)=(p-1)(q-1)=16\times22=352$

\item Nous connaissons la relation suivante: $E.D\equiv 1[\varphi(N)]$.\\
Cette relation peut être vérifiée dans notre cas:\\ $151\times7\equiv 1[352]\leftrightarrow151\times7[352]= 1$ \textit{(vérifié avec le programme cf \ref{exo1})}
\end{enumerate}
\end{exercice}

\begin{exercice}
\begin{enumerate}
\item $N=221$, $E=11$ et $D=35$
\begin{enumerate}
\item Soit $M=112$ le message et $C$ le message crypté, alors:\\
$C=M^{E}[N]=112^{11}[221]=122$
\item Soit $C=78$ le message reçu et $m$ le message originel, alors:\\
$M=C^{D}[N]=78^{35}[221]=65$
\end{enumerate}
\item $p=53$, $q=71$
\begin{enumerate}
\item $N=53\times 71=3763$\\$\varphi(N)=52\times 70=3640$
\item $E=307$: $E<\varphi(N) \land pgcd(\varphi(N),E)=1$
\begin{itemize}
\item 307<3640
\item $pgcd(\varphi(N),E)=3640\times (-7)+307\times 83=1$
\end{itemize}
$E$ est donc acceptable.\\
$D=E^{-1}[\varphi(N)]=83$
\item
\begin{itemize}
\item Clé publique: $E=307$ et $N=3763$
\item Clé privée: $D=83$ et $N$ \textit{(déjà connu avec la clé publique)}
\end{itemize}
\item Les éléments restants sont $p$ et $q$. Étant les générateurs de $N$, ils doivent être dissimulés car ils sont de fait les détenteurs de l'asymétrie du code RSA. Pour rappel, afin de pouvoir décoder et lire RSA, il faut posséder $D$ qui est l'inverse modulaire de $E$ modulo $(p-1)(q-1)$. Le processus pour déterminer $p$ et $q$ à partir de $N$ est gourmand en ressources (cf. \ref{exo1}). Ainsi, plus $p$ et $q$ seront grands, plus le décodage par \textit{"brute force"} demandera de ressources temporelles ou spatiales.
\end{enumerate}
\end{enumerate}
\end{exercice}
\begin{exercice}
\\$E=257$, $N=1073$, $D=353$.
\begin{enumerate}
\item Chiffrer "METHODE":\\
Correspond à $12;04;19;07;14;03;04$.\\
Soit regroupé par paquet de 3: $120;419;071;403;04$.
\begin{itemize}
\item $120^{257}[1073]=589$
\item $419^{257}[1073]=673$
\item $71^{257}[1073]=238$
\item $403^{257}[1073]=308$
\item $4^{257}[1073]=1024$
\end{itemize}
Nous avons donc: $589;673;238;308;1024$.

\item Déchiffrer $263;115;613;10$:
\begin{itemize}
\item $263^{353}[1073]=21$
\item $115^{353}[1073]=724$
\item $613^{353}[1073]=151$
\item $10^{353}[1073]=914$
\end{itemize}
Nous avons donc: $21;724;151;914$.\\
Soit regroupé par paquet de 2: $21;07;24;15;19;14$.\\
Le message est: "CRYPTO".

\item Chiffrer "AVEZVOUSBIENREUSSI":\\
Correspond à $00;21;04;25;21;14;20;18;01;08;04;13;17;04;20;18;18;08$.\\
Soit regroupé par paquet de 3: $002;104;252;114;201;801;080;413;170;$ $420;181;808$.
\begin{itemize}
\item $2^{257}[1073]=32$
\item $104^{257}[1073]=916$
\item $252^{257}[1073]=546$
\item $114^{257}[1073]=983$
\item $201^{257}[1073]=403$
\item $801^{257}[1073]=1001$
\item $80^{257}[1073]=709$
\item $413^{257}[1073]=857$
\item $170^{257}[1073]=716$
\item $420^{257}[1073]=1034$
\item $181^{257}[1073]=567$
\item $808^{257}[1073]=919$
\end{itemize}
Nous avons donc: $32;916;546;983;403;1001;709;857;716;1034;567;919$.

\item Déchiffrer $1019;35;567;36;384;703;99;59$:
\begin{itemize}
\item $1019^{353}[1073]=180$
\item $35^{353}[1073]=13$
\item $567^{353}[1073]=181$
\item $36^{353}[1073]=517$
\item $384^{353}[1073]=140$
\item $703^{353}[1073]=111$
\item $99^{353}[1073]=41$
\item $59^{353}[1073]=204$
\end{itemize}
Nous avons donc: $180;13;181;517;140;111;41;204$.\\
Soit regroupé par paquet de 2: $18;00;13;18;15;17;14;01;11;04;12;04$.\\
Le message est: "SANSPROBLEME".

\item Déchiffrer $553;813$:
\begin{itemize}
\item $553^{353}[1073]=36$
\item $813^{353}[1073]=813$

\end{itemize}
Nous avons donc: $36;813$.\\
Soit regroupé par paquet de 2: $03;06;08;13$.\\
Le message est: "DGIN".
\end{enumerate}
\end{exercice}
\section{Annexes}
\subsection{Programme solution de l'exercice 1}\label{exo1}
\begin{lstlisting}
import math
def erathosthene(n):
    t=[]
    r=[]
    t+=[False]
    t+=[False]
    for i in range(2,n):
        t+=[True]
    for i in range(2,int(math.sqrt(n))):
        j=2*i
        while j<len(t):
            t[j]=False
            j=j+i
    for i in range(2,n):
        if t[i]:
            r+=[i]
    return r
def scan(tab,n):
    for i in range(len(t)):
        for y in range(len(t)):
            if tab[i]*tab[y]==n:
                return [tab[i],tab[y]]
    return False

e=151
d=7
n=391

print("e=",e,", d=",d,", n=",n,sep="")
print("====================")
t=erathosthene(300)
t=scan(t,n)
print("p=",t[0]," et q=",t[1],sep="")
phi=(t[0]-1)*(t[1]-1)
print("phi(N) =",phi)
print("E*D%phi =",e*d%phi)
\end{lstlisting}
\end{document}