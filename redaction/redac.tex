\documentclass[10pt,a4paper,french]{article}
\author{par Léo Peyronnet}
\title{Code RSA}
\date{Décembre 2022}

\usepackage[utf8]{inputenc}
\usepackage[T1]{fontenc}

\usepackage{babel}
\usepackage{listings}
\usepackage{amsfonts}
\usepackage{amsmath}
\usepackage{amssymb}
\usepackage{amsthm}
\usepackage{mathtools}
\usepackage{tabto}
\usepackage{color}

\newcounter{exercice}[section]
\newenvironment{exercice}[1][]{\refstepcounter{exercice}\par\medskip
   \noindent \textbf{Exercice~\theexercice #1} \rmfamily}{\medskip}


\begin{document}
\maketitle
\begin{exercice}
\begin{itemize}
\item Clé publique: $N=391$ et $E=151$
\item Clé privée: $D=7$
\end{itemize}
\begin{enumerate}
\item Message reçu et crypté: $C=17$\\Soit $M$ le message tel qu'envoyé (non crypté), alors:\\ $M=C^{D}[N]=17^{7}[391]=204$
\end{enumerate}
\end{exercice}

\end{document}